\documentclass[12pt]{article}
%\usepackage{fullpage}
\usepackage{epic}
\usepackage{eepic}
\usepackage{paralist}
\usepackage{graphicx}
\usepackage{algorithm,algorithmic}
\usepackage{tikz}
\usepackage{xcolor,colortbl}
\usepackage{amsmath, amssymb}

%%%%%%%%%%%%%%%%%%%%%%%%%%%%%%%%%%%%%%%%%%%%%%%%%%%%%%%%%%%%%%%%
% This is FULLPAGE.STY by H.Partl, Version 2 as of 15 Dec 1988.
% Document Style Option to fill the paper just like Plain TeX.

\typeout{Style Option FULLPAGE Version 2 as of 15 Dec 1988}

\topmargin 0pt
\advance \topmargin by -\headheight
\advance \topmargin by -\headsep

\textheight 8.9in

\oddsidemargin 0pt
\evensidemargin \oddsidemargin
\marginparwidth 0.5in

\textwidth 6.5in
%%%%%%%%%%%%%%%%%%%%%%%%%%%%%%%%%%%%%%%%%%%%%%%%%%%%%%%%%%%%%%%%

\pagestyle{empty}
\setlength{\oddsidemargin}{0in}
\setlength{\topmargin}{-0.8in}
\setlength{\textwidth}{6.8in}
\setlength{\textheight}{9.5in}

\setcounter{secnumdepth}{0}

\setlength{\parindent}{0in}
\addtolength{\parskip}{0.2cm}
\setlength{\fboxrule}{.5mm}\setlength{\fboxsep}{1.2mm}
\newlength{\boxlength}\setlength{\boxlength}{\textwidth}
\addtolength{\boxlength}{-4mm}

\newcommand{\algosolutionbox}[2]{
  \begin{center}
    \framebox{\parbox{\boxlength}{
        \textbf{CS 5722, Fall 2014} \hfill \textbf{#1}\\
        #2
      }}
  \end{center}}


\begin{document}

\algosolutionbox{Homework 2}{
  % TODO: fill in your own name, netID, and collaborators
  Group: Michael Jalkio, Kevin Li, Daniel Sperling\\
  NetIDs: mrj77, kyl27, dhs252\\
}

\bigskip


\subsection{1}
\subsubsection{(a)}
We know that $MaxCost - MinCost = 100$ and assume that the distribution of costs is uniform between $MaxCost$ and $MinCost$.  Using method 1 this means:
\begin{align*}
Avg\Delta Cost = 0.25(MaxCost-MinCost)=0.25*100=25
\end{align*}
If we want $P_{initial}=0.4$ we should set our $T_0$ such that $e^{-Avg\Delta Cost / T_0} = 0.4$:
\begin{align*}
e^{-Avg\Delta Cost / T_0} & = 0.4\\
\frac{-25}{T_0} & = ln(0.4)\\
T_0 & = \frac{-25}{ln(0.4)} \approx 27.28
\end{align*}

\subsubsection{(b)}
To find a general expression for $T_0$ in terms of $MaxCost$, $MinCost$, and $P_1$:
\begin{align*}
Avg\Delta Cost & = 0.25(MaxCost-MinCost)\\\\
P_1 & = e^{-Avg\Delta Cost / T_0} = e^{-0.25(MaxCost-MinCost) / T_0}\\\\
\therefore T_0 & = \frac{-0.25(MaxCost-MinCost)}{ln(P_1)}
\end{align*}

\subsubsection{(c)}
For $T_f$ (a better name for $Tfinal$) nothing changes.  The relationship between $MaxCost$, $MinCost$, and the probability of an uphill move are exactly the same.
\begin{align*}
T_f & = \frac{-0.25(MaxCost-MinCost)}{ln(P_2)}
\end{align*}

\subsubsection{(d)}
We have a simulated annealing algorithm with $T_0 = 100$, $Maxtime = 200$, $\beta=1$, $M=1$, and $MaxCost - MinCost = 100$.  We must find the correct value of $\alpha$ so the probability of moving uphill on the 200th iteration is $0.001$.

We found the equation for this in lecture:
\begin{align*}
\alpha  = \left( \frac{-Avg\Delta Cost}{T_0 ln(P_2)} \right)^{1/G}
= \left( \frac{-25}{100 ln(0.001)} \right)^{1/200}
\approx 0.984
\end{align*}

\subsubsection{(e)}
The $M>1$ case was also discussed in lecture, so if $M=10$:
\begin{align*}
\alpha  = \left( \frac{-Avg\Delta Cost}{T_0 ln(P_2)} \right)^{M/GM}
= \left( \frac{-25}{100 ln(0.001)} \right)^{10/200}
= \left( \frac{-1}{4 ln(0.001)} \right)^{1/20}
\approx 0.847
\end{align*}
Intuitively, if $M$ is bigger this means that the temperature will be reduced fewer times, so each of those times we need to decrease it more.

\subsection{2}
Given that the the lowest $Cost(j)$ in the range is when $j = 1$, as $Cost(1) = 40$, $S_0 = 1$ would be chosen as the initial value of $S$. 

As all the remaining values of $S$ chosen are neighbors of $S = 1$, the average change in cost is the average of the difference between $Cost(j)$ when $j = 2,3,4,5,6$ and $Cost(1)$. The differences in costs are $20$, $10$, $25$, $35$, and $5$, the average change in cost is $(20 + 10 + 25 + 35 + 5) / 5 = 19$. 

As discussed in lecture, $T_0$ can be estimated ignoring both the value for $\alpha$ and $M$ - it can depend purely on $P_1$ and the $Avg\Delta Cost$ using the estimation function $T_0 = -Avg\Delta Cost / ln(P_1)$. Given the value of $19$ for $Avg\Delta Cost$ and $0.9$ for $P_1$, $T_0 = -19/ln(0.9) \approx 180.333$. 

\end{document}
