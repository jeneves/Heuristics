\documentclass[12pt]{article}
%\usepackage{fullpage}
\usepackage{epic}
\usepackage{eepic}
\usepackage{paralist}
\usepackage{graphicx}
\usepackage{algorithm,algorithmic}
\usepackage{tikz}
\usepackage{xcolor,colortbl}
\usepackage{amsmath, amssymb}

%%%%%%%%%%%%%%%%%%%%%%%%%%%%%%%%%%%%%%%%%%%%%%%%%%%%%%%%%%%%%%%%
% This is FULLPAGE.STY by H.Partl, Version 2 as of 15 Dec 1988.
% Document Style Option to fill the paper just like Plain TeX.

\typeout{Style Option FULLPAGE Version 2 as of 15 Dec 1988}

\topmargin 0pt
\advance \topmargin by -\headheight
\advance \topmargin by -\headsep

\textheight 8.9in

\oddsidemargin 0pt
\evensidemargin \oddsidemargin
\marginparwidth 0.5in

\textwidth 6.5in
%%%%%%%%%%%%%%%%%%%%%%%%%%%%%%%%%%%%%%%%%%%%%%%%%%%%%%%%%%%%%%%%

\pagestyle{empty}
\setlength{\oddsidemargin}{0in}
\setlength{\topmargin}{-0.8in}
\setlength{\textwidth}{6.8in}
\setlength{\textheight}{9.5in}

\setcounter{secnumdepth}{0}

\setlength{\parindent}{0in}
\addtolength{\parskip}{0.2cm}
\setlength{\fboxrule}{.5mm}\setlength{\fboxsep}{1.2mm}
\newlength{\boxlength}\setlength{\boxlength}{\textwidth}
\addtolength{\boxlength}{-4mm}

\newcommand{\algosolutionbox}[2]{
  \begin{center}
    \framebox{\parbox{\boxlength}{
        \textbf{CS 5722, Fall 2014} \hfill \textbf{#1}\\
        #2
      }}
  \end{center}}


\begin{document}

\algosolutionbox{Homework 4}{
  Group: Michael Jalkio, Kevin Li, Daniel Sperling\\
  NetIDs: mrj77, kyl27, dhs252
}

\subsection{1(a)}
We need a binary string of length 5 for this problem.  As we learned in class, a binary string of length $n$ can be used to represent integers from 0 to $2^n - 1$.  Here we need to represent the numbers from 0 to 31, so if we plug in $n=5$ we see that a binary string of length five can be used for the integer range $[0,31]$.

\subsubsection{(b)}
I would define my neighborhood as all binary strings with a single bit changed.  Since there are 5 bits that we can flip in our binary string, this means that our neighborhood has 5 members.  All the neighbors are guaranteed to be valid options because every binary string of length 5 maps to an base 10 integer between 0 and 31.

\subsection{2(a)}
The decision vector will be represented as an integer vector of length 10. The $i$-th bit in the vector will represent the chip that the $i$-th cell is in. For instance, consider the vector: $[1, 1, 1, 1, 1, 2, 2, 2, 2, 2]$. Cells 1-5 are on the first chip, and cells 6-10 are on the second chip.

\subsubsection{(b)}
The neighborhood for a given decision vector will be the set of vectors where one pair of different-valued elements are swapped.  Example neighbors:\\
$[2, 1, 1, 1, 1, 2, 2, 2, 2, 1]$ (swap cell 1 with cell 10)\\
$[1, 1, 2, 1, 1, 2, 2, 1, 2, 2]$  (swap cell 3 with cell 8)\\\\
There are exactly $5 * 5 = 25$ neighbors because each of the 5 cells on chip 1 can be swapped with each of the 5 cells on chip 2.

\subsubsection{(c)}
Iteration 1 (initial) : $[1, 1, 1, 1, 1, 2, 2, 2, 2, 2]$

Iteration 2 : $[1, 1, 2, 1, 1, 1, 2, 2, 2, 2]$  (swap cell 3 with cell 6)

Iteration 3 : $[2, 1, 2, 1, 1, 1, 2, 2, 1, 2]$  (swap cell 1 with cell 9)

Tabu Neighbors for Iteration 4:\\
$[2, 1, 1, 1, 1, 2, 2, 2, 1, 2]$  (cannot swap cells 3 and 6)\\
$[1, 1, 2, 1, 1, 1, 2, 2, 2, 2]$  (cannot swap cells 2 and 9)

\end{document}
